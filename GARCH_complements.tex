\documentclass[8pt]{jsarticle}
\usepackage{amsmath}
\usepackage{amssymb}
\usepackage{amsfonts}
\usepackage{latexsym}
\usepackage{ascmac}
\allowdisplaybreaks[1]
\newcommand{\Section}[2]{\section*{\S #1 .\hspace{5pt} #2}}
\newtheorem{prop}{定理}
\newtheorem{proof}{証明}
\def\qed{\hfill $\Box$}
\def\vector#1{\mbox{\boldmath $#1$}}
\begin{document}

\Section{1}{GENERAKIZED\ AUTOREGRESSIVE\ CONDITIONAL\ HETEROSLEDASTICITY}

$GARCH$モデル:
\begin{eqnarray*}
	h_t &=& \alpha_0 + \sum_{i=1}^{q} \alpha_i \epsilon_{t-i}^2 + \sum_{i=1}^{p} \beta_i h_{t-i} \\
	&=& \alpha_0 + A(L) \epsilon_t^2 + B(L) h_t.
\end{eqnarray*}

\begin{boxnote}
	\begin{prop}
	$GARCH(p,\ q)$ 過程について,広義定常である為の必要十分条件は$A(1)+B(1) < 1$である:
	\[
		E[\epsilon_t]=0,\ V[\epsilon_t]=\alpha_0(1-A(1)-B(1))^{-1},\ Cov[\epsilon_t,\ \epsilon_s]=0
		\hspace{20pt} \Leftrightarrow \hspace{20pt}
		A(1)+B(1) < 1.
	\]
	\end{prop}
\end{boxnote}
\begin{proof}
$\epsilon_t$は次の様に表されると仮定する:
\[
	\epsilon_t \equiv \eta_t h_t^{\frac{1}{2}}, \hspace{20pt} \eta_t \sim N(0,\ 1)\ i.i.d.
\]
各時点の$\epsilon_*$を繰り返し$h_t$に代入する:
\begin{align*}
	h_t &= \alpha_0 + \sum_{i=1}^{q} \alpha_i \epsilon_{t-i}^2 + \sum_{i=1}^{p} \beta_i h_{t-i} \\
	&= \alpha_0 + \sum_{i=1}^{q} \alpha_i \eta_{t-i}^2 h_{t-i} + \sum_{i=1}^{p} \beta_i h_{t-i} \\
	&= \alpha_0 + \sum_{i=1}^{q} \alpha_i \eta_{t-i}^2 \left( \alpha_0 + \sum_{j=1}^{q} \alpha_j \eta_{t-i-j}^2 h_{t-i-j} + \sum_{j=1}^{p} \beta_j h_{t-i-j} \right)\\
		&\quad+ \sum_{i=1}^{p} \beta_i \left( \alpha_0 + \sum_{j=1}^{q} \alpha_j \eta_{t-i-j}^2 h_{t-i-j} + \sum_{j=1}^{p} \beta_j h_{t-i-j} \right) \\
	&= \alpha_0 + \alpha_0 \left( \sum_{i=1}^{q} \alpha_i \eta_{t-i}^2 + \sum_{i=1}^{p} \beta_i \right) \\
		&\quad+ \left( \sum_{i=1}^{q} \alpha_i \eta_{t-i}^2 + \sum_{i=1}^{p} \beta_i \right) \sum_{j=1}^{q} \alpha_j \eta_{t-i-j}^2 h_{t-i-j} \\
		&\quad+ \left( \sum_{i=1}^{q} \alpha_i \eta_{t-i}^2 + \sum_{i=1}^{p} \beta_i \right) \sum_{j=1}^{p} \beta_j h_{t-i-j} \\
	&= \alpha_0 + \alpha_0 \left( \sum_{i=1}^{q} \alpha_i \eta_{t-i}^2 + \sum_{i=1}^{p} \beta_i \right) \\
		&\quad+ \left( \sum_{i=1}^{q} \alpha_i \eta_{t-i}^2 + \sum_{i=1}^{p} \beta_i \right) \sum_{j=1}^{q} \alpha_j \eta_{t-i-j}^2 \left( \alpha_0 + \sum_{l=1}^{q} \alpha_l \eta_{t-i-j-l}^2 h_{t-i-j-l} + \sum_{l=1}^{p} \beta_l h_{t-i-j-l} \right) \\
		&\quad+ \left( \sum_{i=1}^{q} \alpha_i \eta_{t-i}^2 + \sum_{i=1}^{p} \beta_i \right) \sum_{j=1}^{p} \beta_j \left( \alpha_0 + \sum_{l=1}^{q} \alpha_l \eta_{t-i-j-l}^2 h_{t-i-j-l} + \sum_{l=1}^{p} \beta_l h_{t-i-j-l} \right) \\
	&= \alpha_0 + \alpha_0 \left( \sum_{i=1}^{q} \alpha_i \eta_{t-i}^2 + \sum_{i=1}^{p} \beta_i \right) + \alpha_0 \left( \left( \sum_{j=1}^{q} \alpha_j \eta_{t-i-j}^2 + \sum_{j=1}^{p} \beta_j \right)\sum_{i=1}^{q} \alpha_i \eta_{t-i}^2 + \left( \sum_{j=1}^{q} \alpha_j \eta_{t-i-j}^2 + \sum_{j=1}^{p} \beta_j \right)\sum_{i=1}^{p} \beta_i \right) \\
		&\quad+ \left( \left( \sum_{j=1}^{q} \alpha_j \eta_{t-i-j}^2 + \sum_{j=1}^{p} \beta_j \right)\sum_{i=1}^{q} \alpha_i \eta_{t-i}^2 
		+ \left( \sum_{j=1}^{q} \alpha_j \eta_{t-i-j}^2 + \sum_{j=1}^{p} \beta_j \right)\sum_{i=1}^{p} \beta_i \right)
		\sum_{l=1}^{q} \alpha_l \eta_{t-i-j-l}^2 h_{t-i-j-l} \\
		&\quad+ \left( \left( \sum_{j=1}^{q} \alpha_j \eta_{t-i-j}^2 + \sum_{j=1}^{p} \beta_j \right)\sum_{i=1}^{q} \alpha_i \eta_{t-i}^2 
		+ \left( \sum_{j=1}^{q} \alpha_j \eta_{t-i-j}^2 + \sum_{j=1}^{p} \beta_j \right)\sum_{i=1}^{p} \beta_i \right)
		\sum_{l=1}^{p} \beta_l h_{t-i-j-l} \\
	&= \alpha_0  M(t,\ 0) + \alpha_0  M(t,\ 1) + \alpha_0  M(t,\ 2) \\
		&\quad+ M(t,\ 2) \sum_{l=1}^{q} \alpha_l \eta_{t-i-j-l}^2 \left( \alpha_0 + \sum_{r=1}^{q} \alpha_r \eta_{t-i-j-l-r}^2 h_{t-i-j-l-r} + \sum_{r=1}^{p} \beta_r h_{t-i-j-l-r} \right) \\
		&\quad+ M(t,\ 2) \sum_{l=1}^{p} \beta_l \left( \alpha_0 + \sum_{r=1}^{q} \alpha_r \eta_{t-i-j-l-r}^2 h_{t-i-j-l-r} + \sum_{r=1}^{p} \beta_r h_{t-i-j-l-r} \right) \\
	&\vdots \\
	&= \alpha_0 \sum_{k=0}^{\infty} M(t,\ k).
\end{align*}
上式で記述されている通り,$M(t,\ k)$は以下の規則で表される:
\begin{align*}
	M(t,\ 0) &= 1, \\
	M(t,\ 1) &= \left( \sum_{i=1}^{q} \alpha_i \eta_{t-i}^2 + \sum_{i=1}^{p} \beta_i \right), \\
	M(t,\ 2) &= \left( \left( \sum_{j=1}^{q} \alpha_j \eta_{t-i-j}^2 + \sum_{j=1}^{p} \beta_j \right)\sum_{i=1}^{q} \alpha_i \eta_{t-i}^2 + \left( \sum_{j=1}^{q} \alpha_j \eta_{t-i-j}^2 + \sum_{j=1}^{p} \beta_j \right)\sum_{i=1}^{p} \beta_i \right) \\
		&= M(t-i,\ 1) \sum_{i=1}^{q} \alpha_i \eta_{t-i}^2 + M(t-i,\ 1) \sum_{i=1}^{p} \beta_i \\
	M(t,\ 3) &= M(t,\ 2) \sum_{l=1}^{q} \alpha_l \eta_{t-i-j-l}^2 + M(t,\ 2) \sum_{l=1}^{p} \beta_l \\
		&= \left( \left( \sum_{j=1}^{q} \alpha_j \eta_{t-i-j}^2 + \sum_{j=1}^{p} \beta_j \right)\sum_{i=1}^{q} \alpha_i \eta_{t-i}^2 + \left( \sum_{j=1}^{q} \alpha_j \eta_{t-i-j}^2 + \sum_{j=1}^{p} \beta_j \right)\sum_{i=1}^{p} \beta_i \right) \sum_{l=1}^{q} \alpha_l \eta_{t-i-j-l}^2 \\
			&\quad+ \left( \left( \sum_{j=1}^{q} \alpha_j \eta_{t-i-j}^2 + \sum_{j=1}^{p} \beta_j \right)\sum_{i=1}^{q} \alpha_i \eta_{t-i}^2 + \left( \sum_{j=1}^{q} \alpha_j \eta_{t-i-j}^2 + \sum_{j=1}^{p} \beta_j \right)\sum_{i=1}^{p} \beta_i \right) \sum_{l=1}^{p} \beta_l \\
		&= \sum_{i=1}^{q} \alpha_i \eta_{t-i}^2 \left( \left( \sum_{j=1}^{q} \alpha_j \eta_{t-i-j}^2 + \sum_{j=1}^{p} \beta_j \right) \left( \sum_{l=1}^{q} \alpha_l \eta_{t-i-j-l}^2 + \sum_{l=1}^{p} \beta_l \right) \right) \\
			&\quad+ \sum_{i=1}^{p} \beta_i \left( \left( \sum_{j=1}^{q} \alpha_j \eta_{t-i-j}^2 + \sum_{j=1}^{p} \beta_j \right) \left( \sum_{l=1}^{q} \alpha_l \eta_{t-i-j-l}^2 + \sum_{l=1}^{p} \beta_l \right) \right) \\
		&= M(t-i,\ 2) \sum_{i=1}^{q} \alpha_i \eta_{t-i}^2 + M(t-i,\ 2) \sum_{i=1}^{p} \beta_i \\
	\vdots \\
	M(t,\ k+1) &= M(t-i,\ k) \sum_{i=1}^{q} \alpha_i \eta_{t-i}^2 + M(t-i,\ k) \sum_{i=1}^{p} \beta_i \\
	\vdots	
\end{align*}
$\eta_t$は独立に分布$N(0,\ 1)$に従うという仮定の下,$M(t,\ k)$のモーメントは時点$t$に依存しない:
\[
	E[M(t,\ k)^r] = E[M(s,\ k)^r], \hspace{20pt} t \neq s,\ r \in \mathbb{N}.
\]
従って$1$次モーメントの計算は次の結果を得る.$k \geq 1$として,
\begin{align*}
	E[M(t,\ k)] &= E\left[ M(t-i,\ k-1) \sum_{i=1}^{q} \alpha_i \eta_{t-i}^2 + M(t-i,\ k-1) \sum_{i=1}^{p} \beta_i \right] \\
	&= \left( \sum_{i=1}^{q} \alpha_i E[\eta_{t-i}^2] + \sum_{i=1}^{p} \beta_i \right) E[M(t,\ k-1)] \\
	&= \left( \sum_{i=1}^{q} \alpha_i + \sum_{i=1}^{p} \beta_i \right)^2 E[M(t,\ k-2)] \\
	&\vdots \\
	&= \left( \sum_{i=1}^{q} \alpha_i + \sum_{i=1}^{p} \beta_i \right)^k E[M(t,\ 0)] \\
	&= \left( \sum_{i=1}^{q} \alpha_i + \sum_{i=1}^{p} \beta_i \right)^k.
\end{align*}
最終的に$\epsilon_t^2$の$1$次モーメントは,
\begin{align*}
	E[\epsilon_t^2\ |\ \psi_{t-1}] = E[h_t] &= E\left[ \alpha_0 \sum_{k=0}^{\infty} M(t,\ k) \right] 
	= \alpha_0 E\left[ \sum_{k=0}^{\infty} M(t,\ k) \right]
\end{align*}
ここで定理の広義定常性の仮定から$E[h_t] < \infty$でなければならず,従って$\sum_{k=0}^{\infty} M(t,\ k)\ (\geq 0)$は可積分である必要がある.
よって$\sum_{i=1}^{q} \alpha_i + \sum_{i=1}^{p} \beta_i < 1$である.逆に,
$\sum_{i=1}^{q} \alpha_i + \sum_{i=1}^{p} \beta_i < 1$が仮定されていれば無限級数が可積分である為,
\[
	\alpha_0 E\left[ \sum_{k=0}^{\infty} M(t,\ k) \right] = \alpha_0 \left( 1 - \sum_{i=1}^{q} \alpha_i - \sum_{i=1}^{p} \beta_i \right)^{-1} < \infty
\]
が正当化される.そしてこれは時点に依存しない分散である.また,
\begin{align*}
	E[\epsilon_t\ |\ \psi_{t-1}] &= E[\eta_t]E[h_t^\frac{1}{2}] = 0,\\
	Cov[\epsilon_t,\ \epsilon_s] &= Cov[\eta_t h_t^\frac{1}{2},\ \eta_s h_s^\frac{1}{2}] \\
	&= E[\eta_t h_t^\frac{1}{2}\ \eta_s h_s^\frac{1}{2}] \\
	&= 0.
\end{align*}
であることも従うので,定理が示されるのである.
\qed
\end{proof}

\begin{boxnote}
	\begin{prop}
	hoge
	\end{prop}
\end{boxnote}
\begin{proof}
\end{proof}

\begin{boxnote}
	\begin{prop}
	
	\end{prop}
\end{boxnote}
\begin{proof}
\end{proof}


\end{document}