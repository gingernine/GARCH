\documentclass[8pt]{jsarticle}
\usepackage{amsmath}
\usepackage{amssymb}
\usepackage{amsfonts}
\usepackage{latexsym}
\usepackage{ascmac}
\allowdisplaybreaks[1]
\newcommand{\Section}[2]{\section*{\S #1 .\hspace{5pt} #2}}
\newtheorem{prop}{定理}
\newtheorem{proof}{証明}
\def\qed{\hfill $\Box$}
\def\vector#1{\mbox{\boldmath $#1$}}
\begin{document}

$本稿は GARCH 理論の習得の覚書である.論文には理論部分の証明が Appendix として載せられている.
その証明を読まないとモデルが何を表しているのかわからないので自分でも読んで書いて覚えようとするのだけれども,省略されている部分を懇切丁寧に補ってもらわないと
僕が付いて行けないので,後学のために証明にくどいほど補間して書き直しておくのである.$

\Section{1}{GENERALIZED\ AUTOREGRESSIVE\ CONDITIONAL\ HETEROSKEDASTICITY}

$GARCH$モデル:
\begin{eqnarray*}
	&h_t = \alpha_0 + \sum_{i=1}^{q} \alpha_i \epsilon_{t-i}^2 + \sum_{i=1}^{p} \beta_i h_{t-i}, \\
	&\alpha_0 \geq 0, \hspace{20pt} \alpha_i > 0, \hspace{20pt} \beta_i \geq 0.
\end{eqnarray*}

\begin{boxnote}
	\begin{prop}
	$GARCH(p,\ q)$ 過程について,広義定常である為の必要十分条件は$\sum_{i=1}^{q} \alpha_i + \sum_{i=1}^{p} \beta_i < 1$である:
	\[
		E[\epsilon_t]=0,\ V[\epsilon_t]=\alpha_0 \left( 1 - \sum_{i=1}^{q} \alpha_i - \sum_{i=1}^{p} \beta_i \right)^{-1},\ Cov[\epsilon_t,\ \epsilon_s]=0
		\hspace{20pt} \Leftrightarrow \hspace{20pt}
		\sum_{i=1}^{q} \alpha_i + \sum_{i=1}^{p} \beta_i < 1.
	\]
	\end{prop}
\end{boxnote}
\begin{proof}
$\epsilon_t$は次の様に表されると仮定する:
\[
	\epsilon_t \equiv \eta_t h_t^{\frac{1}{2}}, \hspace{20pt} \eta_t \sim N(0,\ 1)\ i.i.d.
\]
各時点の$\epsilon_*$を繰り返し$h_t$に代入する:
\begin{align*}
	h_t &= \alpha_0 + \sum_{i=1}^{q} \alpha_i \epsilon_{t-i}^2 + \sum_{i=1}^{p} \beta_i h_{t-i} \\
	&= \alpha_0 + \sum_{i=1}^{q} \alpha_i \eta_{t-i}^2 h_{t-i} + \sum_{i=1}^{p} \beta_i h_{t-i} \\
	&= \alpha_0 + \sum_{i=1}^{q} \alpha_i \eta_{t-i}^2 \left( \alpha_0 + \sum_{j=1}^{q} \alpha_j \eta_{t-i-j}^2 h_{t-i-j} + \sum_{j=1}^{p} \beta_j h_{t-i-j} \right)\\
		&\quad+ \sum_{i=1}^{p} \beta_i \left( \alpha_0 + \sum_{j=1}^{q} \alpha_j \eta_{t-i-j}^2 h_{t-i-j} + \sum_{j=1}^{p} \beta_j h_{t-i-j} \right) \\
	&= \alpha_0 + \alpha_0 \left( \sum_{i=1}^{q} \alpha_i \eta_{t-i}^2 + \sum_{i=1}^{p} \beta_i \right) \\
		&\quad+ \left( \sum_{i=1}^{q} \alpha_i \eta_{t-i}^2 + \sum_{i=1}^{p} \beta_i \right) \sum_{j=1}^{q} \alpha_j \eta_{t-i-j}^2 h_{t-i-j} \\
		&\quad+ \left( \sum_{i=1}^{q} \alpha_i \eta_{t-i}^2 + \sum_{i=1}^{p} \beta_i \right) \sum_{j=1}^{p} \beta_j h_{t-i-j} \\
	&= \alpha_0 + \alpha_0 \left( \sum_{i=1}^{q} \alpha_i \eta_{t-i}^2 + \sum_{i=1}^{p} \beta_i \right) \\
		&\quad+ \left( \sum_{i=1}^{q} \alpha_i \eta_{t-i}^2 + \sum_{i=1}^{p} \beta_i \right) \sum_{j=1}^{q} \alpha_j \eta_{t-i-j}^2 \left( \alpha_0 + \sum_{l=1}^{q} \alpha_l \eta_{t-i-j-l}^2 h_{t-i-j-l} + \sum_{l=1}^{p} \beta_l h_{t-i-j-l} \right) \\
		&\quad+ \left( \sum_{i=1}^{q} \alpha_i \eta_{t-i}^2 + \sum_{i=1}^{p} \beta_i \right) \sum_{j=1}^{p} \beta_j \left( \alpha_0 + \sum_{l=1}^{q} \alpha_l \eta_{t-i-j-l}^2 h_{t-i-j-l} + \sum_{l=1}^{p} \beta_l h_{t-i-j-l} \right) \\
	&= \alpha_0 + \alpha_0 \left( \sum_{i=1}^{q} \alpha_i \eta_{t-i}^2 + \sum_{i=1}^{p} \beta_i \right) + \alpha_0 \left( \left( \sum_{j=1}^{q} \alpha_j \eta_{t-i-j}^2 + \sum_{j=1}^{p} \beta_j \right)\sum_{i=1}^{q} \alpha_i \eta_{t-i}^2 + \left( \sum_{j=1}^{q} \alpha_j \eta_{t-i-j}^2 + \sum_{j=1}^{p} \beta_j \right)\sum_{i=1}^{p} \beta_i \right) \\
		&\quad+ \left( \left( \sum_{j=1}^{q} \alpha_j \eta_{t-i-j}^2 + \sum_{j=1}^{p} \beta_j \right)\sum_{i=1}^{q} \alpha_i \eta_{t-i}^2 
		+ \left( \sum_{j=1}^{q} \alpha_j \eta_{t-i-j}^2 + \sum_{j=1}^{p} \beta_j \right)\sum_{i=1}^{p} \beta_i \right)
		\sum_{l=1}^{q} \alpha_l \eta_{t-i-j-l}^2 h_{t-i-j-l} \\
		&\quad+ \left( \left( \sum_{j=1}^{q} \alpha_j \eta_{t-i-j}^2 + \sum_{j=1}^{p} \beta_j \right)\sum_{i=1}^{q} \alpha_i \eta_{t-i}^2 
		+ \left( \sum_{j=1}^{q} \alpha_j \eta_{t-i-j}^2 + \sum_{j=1}^{p} \beta_j \right)\sum_{i=1}^{p} \beta_i \right)
		\sum_{l=1}^{p} \beta_l h_{t-i-j-l} \\
	&= \alpha_0  M(t,\ 0) + \alpha_0  M(t,\ 1) + \alpha_0  M(t,\ 2) \\
		&\quad+ M(t,\ 2) \sum_{l=1}^{q} \alpha_l \eta_{t-i-j-l}^2 \left( \alpha_0 + \sum_{r=1}^{q} \alpha_r \eta_{t-i-j-l-r}^2 h_{t-i-j-l-r} + \sum_{r=1}^{p} \beta_r h_{t-i-j-l-r} \right) \\
		&\quad+ M(t,\ 2) \sum_{l=1}^{p} \beta_l \left( \alpha_0 + \sum_{r=1}^{q} \alpha_r \eta_{t-i-j-l-r}^2 h_{t-i-j-l-r} + \sum_{r=1}^{p} \beta_r h_{t-i-j-l-r} \right) \\
	&\vdots \\
	&= \alpha_0 \sum_{k=0}^{\infty} M(t,\ k).
\end{align*}
上式で記述されている通り,$M(t,\ k)$は以下の規則で表される:
\begin{align*}
	M(t,\ 0) &= 1, \\
	M(t,\ 1) &= \left( \sum_{i=1}^{q} \alpha_i \eta_{t-i}^2 + \sum_{i=1}^{p} \beta_i \right), \\
	M(t,\ 2) &= \left( \left( \sum_{j=1}^{q} \alpha_j \eta_{t-i-j}^2 + \sum_{j=1}^{p} \beta_j \right)\sum_{i=1}^{q} \alpha_i \eta_{t-i}^2 + \left( \sum_{j=1}^{q} \alpha_j \eta_{t-i-j}^2 + \sum_{j=1}^{p} \beta_j \right)\sum_{i=1}^{p} \beta_i \right) \\
		&= M(t-i,\ 1) \sum_{i=1}^{q} \alpha_i \eta_{t-i}^2 + M(t-i,\ 1) \sum_{i=1}^{p} \beta_i \\
	M(t,\ 3) &= M(t,\ 2) \sum_{l=1}^{q} \alpha_l \eta_{t-i-j-l}^2 + M(t,\ 2) \sum_{l=1}^{p} \beta_l \\
		&= \left( \left( \sum_{j=1}^{q} \alpha_j \eta_{t-i-j}^2 + \sum_{j=1}^{p} \beta_j \right)\sum_{i=1}^{q} \alpha_i \eta_{t-i}^2 + \left( \sum_{j=1}^{q} \alpha_j \eta_{t-i-j}^2 + \sum_{j=1}^{p} \beta_j \right)\sum_{i=1}^{p} \beta_i \right) \sum_{l=1}^{q} \alpha_l \eta_{t-i-j-l}^2 \\
			&\quad+ \left( \left( \sum_{j=1}^{q} \alpha_j \eta_{t-i-j}^2 + \sum_{j=1}^{p} \beta_j \right)\sum_{i=1}^{q} \alpha_i \eta_{t-i}^2 + \left( \sum_{j=1}^{q} \alpha_j \eta_{t-i-j}^2 + \sum_{j=1}^{p} \beta_j \right)\sum_{i=1}^{p} \beta_i \right) \sum_{l=1}^{p} \beta_l \\
		&= \sum_{i=1}^{q} \alpha_i \eta_{t-i}^2 \left( \left( \sum_{j=1}^{q} \alpha_j \eta_{t-i-j}^2 + \sum_{j=1}^{p} \beta_j \right) \left( \sum_{l=1}^{q} \alpha_l \eta_{t-i-j-l}^2 + \sum_{l=1}^{p} \beta_l \right) \right) \\
			&\quad+ \sum_{i=1}^{p} \beta_i \left( \left( \sum_{j=1}^{q} \alpha_j \eta_{t-i-j}^2 + \sum_{j=1}^{p} \beta_j \right) \left( \sum_{l=1}^{q} \alpha_l \eta_{t-i-j-l}^2 + \sum_{l=1}^{p} \beta_l \right) \right) \\
		&= M(t-i,\ 2) \sum_{i=1}^{q} \alpha_i \eta_{t-i}^2 + M(t-i,\ 2) \sum_{i=1}^{p} \beta_i \\
	\vdots \\
	M(t,\ k+1) &= M(t-i,\ k) \sum_{i=1}^{q} \alpha_i \eta_{t-i}^2 + M(t-i,\ k) \sum_{i=1}^{p} \beta_i \\
	\vdots	
\end{align*}
$\eta_t$は独立に分布$N(0,\ 1)$に従うという仮定の下,$M(t,\ k)$のモーメントは時点$t$に依存しない:
\[
	E[M(t,\ k)^r] = E[M(s,\ k)^r], \hspace{20pt} t \neq s,\ r \in \mathbb{N}.
\]
従って$1$次モーメントの計算は次の結果を得る.$k \geq 1$として,
\begin{align*}
	E[M(t,\ k)] &= E\left[ M(t-i,\ k-1) \sum_{i=1}^{q} \alpha_i \eta_{t-i}^2 + M(t-i,\ k-1) \sum_{i=1}^{p} \beta_i \right] \\
	&= \left( \sum_{i=1}^{q} \alpha_i E[\eta_{t-i}^2] + \sum_{i=1}^{p} \beta_i \right) E[M(t,\ k-1)] \\
	&= \left( \sum_{i=1}^{q} \alpha_i + \sum_{i=1}^{p} \beta_i \right)^2 E[M(t,\ k-2)] \\
	&\vdots \\
	&= \left( \sum_{i=1}^{q} \alpha_i + \sum_{i=1}^{p} \beta_i \right)^k E[M(t,\ 0)] \\
	&= \left( \sum_{i=1}^{q} \alpha_i + \sum_{i=1}^{p} \beta_i \right)^k.
\end{align*}
最終的に$\epsilon_t^2$の$1$次モーメントは,
\begin{align*}
	E[\epsilon_t^2] = E\left[ E[\epsilon_t^2\ |\ \psi_{t-1}] \right] = E[h_t]
	= E\left[ \alpha_0 \sum_{k=0}^{\infty} M(t,\ k) \right]
	= \alpha_0 E\left[ \sum_{k=0}^{\infty} M(t,\ k) \right].
\end{align*}
ここで定理の広義定常性の仮定から$E[h_t] < \infty$でなければならず,従って$\sum_{k=0}^{\infty} M(t,\ k)\ (\geq 0)$は可積分である必要がある.
正項級数が可積分なら$Lebesgue$の収束定理により項別積分に持ち込むことができ,よって$\sum_{i=1}^{q} \alpha_i + \sum_{i=1}^{p} \beta_i < 1$である.逆に,
$\sum_{i=1}^{q} \alpha_i + \sum_{i=1}^{p} \beta_i < 1$が仮定されていれば無限級数が可積分である為,
\[
	\alpha_0 E\left[ \sum_{k=0}^{\infty} M(t,\ k) \right] = \alpha_0 \sum_{k=0}^{\infty} E\left[ M(t,\ k) \right] = \alpha_0 \left( 1 - \sum_{i=1}^{q} \alpha_i - \sum_{i=1}^{p} \beta_i \right)^{-1} < \infty
\]
が正当化される.そしてこれは時点に依存しない分散である.また,時点 $t-1$ までの情報が与えられた下で $h_t$ は決定されることと,
$\eta_t$が時点に関して独立に標準正規分布に従うことに注意すれば,
\begin{align*}
	E[\epsilon_t\ |\ \psi_{t-1}] &= E[\eta_t]h_t^\frac{1}{2} = 0\ \\
	&\Rightarrow\ E[\epsilon_t] = 0, \\
	Cov[\epsilon_t,\ \epsilon_s\ |\ \psi_{t-1}] &= Cov[\eta_t h_t^\frac{1}{2},\ \eta_s h_s^\frac{1}{2}\ |\ \psi_{t-1}] \\
	&= E[\eta_t h_t^\frac{1}{2}\ \eta_s h_s^\frac{1}{2}\ |\ \psi_{t-1}] \\
	&= 0 \\
	&\Rightarrow\ Cov[\epsilon_t,\ \epsilon_s] = 0.
\end{align*}
であることも従うので,定理が示されるのである.
\qed
\end{proof}

本稿の最初に示した$GARCH$モデルの式の同値形を表記する:
\begin{align*}
	\epsilon_t &\equiv \eta_t h_t^{\frac{1}{2}}, \hspace{20pt} \eta_t \sim N(0,\ 1)\ i.i.d. \mbox{を仮定している下で,}\\
	h_t &= \alpha_0 + \sum_{i=1}^{q} \alpha_i \epsilon_{t-i}^2 + \sum_{i=1}^{p} \beta_i h_{t-i} \\
	&= \alpha_0 + \sum_{i=1}^{q} \alpha_i \epsilon_{t-i}^2 + \sum_{i=1}^{p} \beta_i \epsilon_{t-i}^2 + \sum_{i=1}^{p} \beta_i (h_{t-i} - \epsilon_{t-i}^2) \\
	&= \alpha_0 + \sum_{i=1}^{q} \alpha_i \epsilon_{t-i}^2 + \sum_{i=1}^{p} \beta_i \epsilon_{t-i}^2 + \sum_{i=1}^{p} \beta_i h_{t-i}(1 - \eta_{t-i}^2).
\end{align*}

\begin{boxnote}
	\begin{prop}
	$GARCH(1,\ 1)について,h_t = \alpha_0 + \alpha_1 \epsilon_{t-1}^2 + \beta_1 h_{t-1} 過程の 2m 次のモーメント E[\epsilon_t^{2m}] が存在する為の必要十分条件は$
	\begin{align*}
		\mu(\alpha_1,\ \beta_1,\ m) \equiv \sum_{j=0}^{m} \binom{m}{j} a_j \alpha_1^j \beta_1^{m-j} < 1, \hspace{20pt} a_0 = 1,\ a_j = \prod_{i=1}^{j} (2i-1)
	\end{align*}
	$であり,このとき 2m 次のモーメントは次の様に表せる:$
	\[
		E[\epsilon_t^{2m}] = a_m \left[ \sum_{n=0}^{m-1} a_n^{-1} E[\epsilon_t^{2n}] \alpha_0^{m-n} \binom{m}{m-n} \mu(\alpha_1, \beta_1, n) \right]
		\times \left[ 1-\mu(\alpha_1, \beta_1, m) \right]^{-1}.
	\]
	\end{prop}
\end{boxnote}
\begin{proof}
$標準正規分布に従う確率変数 X の 2m\ (m \in \mathbb{N}) 次のモーメントの計算を確認しておく.$
\begin{align*}
	E[x^{2m}] &= \int_{-\infty}^{\infty} x^{2m} \frac{1}{\sqrt{2 \pi}} e^{-\frac{x^2}{2}} dx \\
	&= 2 \frac{1}{\sqrt{2 \pi}} \int_{0}^{\infty} (2x)^{m} e^{-x} \frac{1}{\sqrt{2x}} dx \\
	&= \frac{1}{\sqrt{2 \pi}} 2^{m+\frac{1}{2}} \Gamma(m+\frac{1}{2}) \\
	&= \frac{1}{\sqrt{2 \pi}} 2^{m+\frac{1}{2}} \frac{(2m-1)(2m-3) \cdots 1}{2^m} \sqrt{\pi} \\
	&= \prod_{i=1}^{m} (2i-1).
\end{align*}
$\eta_t の正規性と,情報構造\psi_{t-1} が与えられた下で h_t が定数となることに注意すると以下の式が成り立つ.$
\begin{align*}
	a_m &\equiv E[\eta_t^{2m}] = \prod_{i=1}^{m} (2i-1), \\
	E[\epsilon_t^{2m}] &= E \left[ E[\epsilon_t^{2m}\ |\ \psi_{t-1}] \right] = E \left[ E[\eta_t^{2m} h_t^m\ |\ \psi_{t-1}] \right] 
	=  E[h_t^m] E[\eta_t^{2m}] = a_m E[h_t^m].
\end{align*}
$GARCH(1,\ 1) の下では h_t = \alpha_0 + \alpha_1 \epsilon_{t-1}^2 + \beta_1 h_{t-1} であるから,$
\begin{align*}
	h_t^m &= \left( \alpha_0 + \alpha_1 \epsilon_{t-1}^2 + \beta_1 h_{t-1} \right)^m \\
	&= \sum_{n=0}{m} \binom{m}{n} \alpha_0^{m-n} \sum_{j=0}{n} \alpha_1^j \beta_1^{n-j} \epsilon_{t-1}^2j h_{t-1}^{n-j}.
\end{align*}
$従って,時点 t-2 までの情報を与えると,$
\begin{align*}
	E[h_t^m\ |\ \psi_{t-2}] &= \sum_{n=0}^{m} \binom{m}{n} \alpha_0^{m-n} \sum_{j=0}^{n} \binom{n}{j} \alpha_1^j \beta_1^{n-j} E[\epsilon_{t-1}^{2j} h_{t-1}^{n-j}\ |\ \psi_{t-2}] \\
	&= \sum_{n=0}^{m} \binom{m}{n} \alpha_0^{m-n} \sum_{j=0}^{n} \binom{n}{j} \alpha_1^j \beta_1^{n-j} h_{t-1}^{n-j} E[\epsilon_{t-1}^{2j}\ |\ \psi_{t-2}] \\
	&= \sum_{n=0}^{m} \binom{m}{n} \alpha_0^{m-n} \sum_{j=0}^{n} \binom{n}{j} \alpha_1^j \beta_1^{n-j} h_{t-1}^{n-j} h_{t-1}^j a_j \\
	&= \sum_{n=0}^{m} \binom{m}{n} \alpha_0^{m-n} \sum_{j=0}^{n} \binom{n}{j} a_j \alpha_1^j \beta_1^{n-j} h_{t-1}^n.
\end{align*}
$ここで m 次元ベクトル \vector{w}_t \equiv (h_t^m, h_t^{m-1}, \cdots, h_t)^{T} に対して,期待値を取ると,$
\begin{align*}
	E[\vector{w}_t\ |\ \psi_{t-2}] &= (\ E[h_t^m\ |\ \psi_{t-2}],\ E[h_t^{m-1}\ |\ \psi_{t-2}],\ E[h_t^{m-2}\ |\ \psi_{t-2}],\ \cdots \ ,\ E[h_t\ |\ \psi_{t-2}]\ )^{T} \\
	&= \left(
		\begin{array}{c}
			\sum\limits_{n=0}^{m} h_{t-1}^n \binom{m}{n} \alpha_0^{m-n} \sum\limits_{j=0}^{n} \binom{n}{j} a_j \alpha_1^j \beta_1^{n-j} \\
			\sum\limits_{n=0}^{m-1} h_{t-1}^n \binom{m-1}{n} \alpha_0^{m-1-n} \sum\limits_{j=0}^{n} \binom{n}{j} a_j \alpha_1^j \beta_1^{n-j} \\
			\sum\limits_{n=0}^{m-2} h_{t-1}^n \binom{m-2}{n} \alpha_0^{m-2-n} \sum\limits_{j=0}^{n} \binom{n}{j} a_j \alpha_1^j \beta_1^{n-j} \\
			\vdots \\
			\sum\limits_{n=0}^{1} h_{t-1}^n \binom{1}{n} \alpha_0^{1-n} \sum\limits_{j=0}^{n} \binom{n}{j} a_j \alpha_1^j \beta_1^{n-j}
		\end{array} 
		\right) \\
	&= a_0\left(
		\begin{array}{c}
			\alpha_0^{m} \\
			\alpha_0^{m-1} \\
			\alpha_0^{m-2} \\
			\vdots \\
			\alpha_0
		\end{array} 
		\right) 
		\\
		&\quad+ {\tiny
		\left(
		\begin{array}{ccccc}
			\sum\limits_{j=0}^{m} \binom{m}{j} a_j \alpha_1^j \beta_1^{m-j} & \binom{m}{m-1} \alpha_0 \sum\limits_{j=0}^{m-1} \binom{m-1}{j} a_j \alpha_1^j \beta_1^{m-1-j} & \binom{m}{m-2} \alpha_0^{2} \sum\limits_{j=0}^{m-2} \binom{m-2}{j} a_j \alpha_1^j \beta_1^{m-2-j} & \ldots & \binom{m}{1} \alpha_0^{m-1} \sum\limits_{j=0}^{1} \binom{1}{j} a_j \alpha_1^j \beta_1^{1-j} \\
			0 & \sum\limits_{j=0}^{m-1} \binom{m-1}{j} a_j \alpha_1^j \beta_1^{m-1-j} & \binom{m-1}{m-2} \alpha_0 \sum\limits_{j=0}^{m-2} \binom{m-2}{j} a_j \alpha_1^j \beta_1^{m-2-j} & \ldots & \binom{m-1}{1} \alpha_0^{m-2} \sum\limits_{j=0}^{1} \binom{1}{j} a_j \alpha_1^j \beta_1^{1-j} \\
			0 & 0 & \sum\limits_{j=0}^{m-2} \binom{m-2}{j} a_j \alpha_1^j \beta_1^{m-2-j} & \ldots & \binom{m-2}{1} \alpha_0^{m-3} \sum\limits_{j=0}^{1} \binom{1}{j} a_j \alpha_1^j \beta_1^{1-j} \\
			\vdots & \vdots & \vdots & \ddots & \vdots \\
			0 & 0 & 0 & \ldots & \sum\limits_{j=0}^{1} \binom{1}{j} a_j \alpha_1^j \beta_1^{1-j}
		\end{array}
		\right)
		}
		\left(
		\begin{array}{c}
			 h_{t-1}^{m} \\
			 h_{t-1}^{m-1} \\
			 h_{t-1}^{m-2} \\
			\vdots \\
			 h_{t-1}
		\end{array}
		\right) \\
	&= \vector{d} + \vector{C} \vector{w}_{t-1}.
\end{align*}
$ここで m \times m 上三角行列 \vector{C} の対角成分に現れる級数を以下の様に特別に表記し直す.上三角行列なので行列式は対角成分のみから計算されるのである.$
\[
	\mu(\alpha_1,\beta_1,\ n) \equiv \sum\limits_{j=0}^{n} \binom{n}{j} a_j \alpha_1^j \beta_1^{n-j}
\]
$また,与えられる情報構造の鮮度を落としていくと,すなわち \psi_{*} の添え字の時点を後退させると以下の式が成り立つ:$
\begin{align*}
	E[\vector{w}_t\ |\ \psi_{t-3}] &= E\left[ E\left[ \vector{w}_t\ |\ \psi_{t-2} \right]\ |\ \psi_{t-3} \right] \\
		&= E\left[ \vector{d} + \vector{C} \vector{w}_{t-1}\ |\ \psi_{t-3} \right] \\
		&= \vector{d} + \vector{C} E\left[ \vector{w}_{t-1}\ |\ \psi_{t-3} \right] \\
		&= \vector{d} + \vector{C} (\vector{d} + \vector{C} \vector{w}_{t-2}) \\
		&= (\vector{I} + \vector{C})\vector{d} + \vector{C}^2 \vector{w}_{t-2}, \\
	E[\vector{w}_t\ |\ \psi_{t-4}] &= E\left[ E\left[ \vector{w}_t\ |\ \psi_{t-3} \right]\ |\ \psi_{t-4} \right] \\
		&= E\left[ (\vector{I} + \vector{C})\vector{d} + \vector{C}^2 \vector{w}_{t-2}\ |\ \psi_{t-4} \right] \\
		&= (\vector{I} + \vector{C})\vector{d} + \vector{C}^2 E\left[ \vector{w}_{t-2}\ |\ \psi_{t-4} \right] \\
		&= (\vector{I} + \vector{C} + \vector{C}^2)\vector{d} + \vector{C}^3 \vector{w}_{t-3}, \\
	\vdots \\
	E[\vector{w}_t\ |\ \psi_{t-k-1}] &= (\vector{I} + \vector{C} + \vector{C}^2 + \cdots + \vector{C}^{k-1})\vector{d} + \vector{C}^k \vector{w}_{t-k}, \\
	\vdots 
\end{align*}
$これは k \to \infty まで考えると,情報が無い下での期待値,即ち E[\vector{w}_t] を表すことになり, E[h_t^m] が E[\vector{w}_t] の第一成分であることから,
行列の冪級数の収束が求める E[\epsilon_t^{2m}] の存在の必要十分条件である:
ところで冪級数の収束の必要十分条件は行列 \vector{C} の全ての固有値が 1 より小さいことである.
そして上三角行列の固有値は全て対角成分 \mu(\alpha_1,\beta_1,\ n) に等しい.このとき \vector{C} のべき乗 \vector{C}^k の対角成分はひたすら
対角成分の累乗になり 0 に収束する為, \vector{C}^k が零行列に収束していくということを注意しておく. GARCH モデルの仮定から  \alpha_1,\beta_1 は非負であり,
a_j は標準正規分布の偶数次のモーメントであるから正数であるから,即ち, \mu(\alpha_1,\beta_1,\ n) は n に関して非負単調増大であり, 
\mu(\alpha_1,\beta_1,\ m) < 1 であることが冪級数の収束の十分条件を為す.よって \mu(\alpha_1,\beta_1,\ m) < 1 であることは E[\epsilon_t^{2m}] の存在の
十分条件となる.一方, E[\epsilon_t^{2m}] が存在すれば 2(m-1),\ 2(m-2),\ \cdots ,\ 2 次モーメントも存在することになるので,$
\[
	E[\vector{w}_t] = \lim_{k \to \infty} E[\vector{w}_t\ |\ \psi_{t-k-1}] < \infty,
\]
$即ち冪級数は収束し,\mu(\alpha_1,\beta_1,\ m) < 1 が従う.$

最後に
\[
	E[\epsilon_t^{2m}] = a_m \left[ \sum_{n=0}^{m-1} a_n^{-1} E[\epsilon_t^{2n}] \alpha_0^{m-n} \binom{m}{m-n} \mu(\alpha_1, \beta_1, n) \right]
		\times \left[ 1-\mu(\alpha_1, \beta_1, m) \right]^{-1}
\]
を示す.
\begin{align*}
	E[\epsilon_t^{2m}] &= a_m E[h_t^m], \\
	E[h_t^m\ |\ \psi_{t-2}] &= \sum_{n=0}^{m} h_{t-1}^n \binom{m}{n} \alpha_0^{m-n} \sum_{j=0}^{n} \binom{n}{j} a_j \alpha_1^j \beta_1^{n-j}
\end{align*}
が成り立っていることにより,
\begin{align*}
	E[\epsilon_t^{2m}] &= a_m E[h_t^m] \\
	&= a_m E\left[ E[h_t^m\ |\ \psi_{t-2}] \right] \\
	&= a_m E\left[ \sum_{n=0}^{m} h_{t-1}^n \binom{m}{n} \alpha_0^{m-n} \sum_{j=0}^{n} \binom{n}{j} a_j \alpha_1^j \beta_1^{n-j} \right] \\
	&= a_m \left[ \sum_{n=0}^{m} E[h_{t-1}^n] \binom{m}{n} \alpha_0^{m-n} \mu(\alpha_1,\beta_1,\ n) \right] \\
	&= a_m \left[ \sum_{n=0}^{m-1} a_n^{-1} E[\epsilon_{t-1}^{2n}] \binom{m}{n} \alpha_0^{m-n} \mu(\alpha_1,\beta_1,\ n) \right] + a_m a_m^{-1}  E[\epsilon_{t-1}^{2m}] \mu(\alpha_1,\beta_1,\ m).
\end{align*}
$E[\epsilon_t^{2m}] が時点に依らない値を持つことに注意しなければならない,というのも,先述の行列の冪級数の収束の箇所で,$
\[
	\left(
	\begin{array}{c}
		a_m^{-1} E[\epsilon_t^{2m}] \\
		a_{m-1}^{-1} E[\epsilon_t^{2(m-1)}] \\
		a_{m-2}^{-1} E[\epsilon_t^{2(m-2)}] \\
		\vdots \\
		a_1^{-1} E[\epsilon_t^2]
	\end{array}
	\right)
	= E[\vector{w}_t] = \lim_{k \to \infty} E[\vector{w}_t\ |\ \psi_{t-k-1}] = (\vector{I} - \vector{C})^{-1} \vector{d}
\]
が成り立っているのである.従って,最右辺第二項を左辺に移せば,求める式
\[
	E[\epsilon_t^{2m}] = a_m \left[ \sum_{n=0}^{m-1} a_n^{-1} E[\epsilon_t^{2n}] \binom{m}{n} \alpha_0^{m-n} \mu(\alpha_1,\beta_1,\ n) \right] [1 - \mu(\alpha_1,\beta_1,\ m)]^{-1}
\]
が導かれる.
\qed
\end{proof}


\end{document}