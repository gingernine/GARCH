\documentclass[slide,10pt]{jsarticle}
\usepackage{amsmath}
\usepackage{amssymb}
\usepackage{amsfonts}
\usepackage{latexsym}
\usepackage{ascmac}
\usepackage[dvipdfmx]{color}
\usepackage{pxfonts}
\def\qed{\hfill $\Box$}
\def\vector#1{\mbox{\boldmath $#1$}}
\def\sheet #1{\section*{\centering \large \bfseries #1}}

\begin{document}

\title{\fontsize{45pt}{2cm}\selectfont \\ゼミ発表\\GARCH モデル\\}
\author{\Huge 学籍番号:201311324 百合川尚学}
\date{\Huge 2016/09/19}
\maketitle

\sheet{\Huge GARCH モデルとは何か}

{\huge
\begin{itemize}
	\vspace{5mm}
	\item $Generalized\ Autoregressive\ Conditional\ Heteroskedastic\ $の頭文字をとったもの.
	\vspace{5mm}
	\item 直訳すると''一般化された 自己回帰 条件付き 不均一分散''の確率過程.
	\vspace{5mm}
	\item $Tim\ Bollerslev$教授によって考案されたモデル.
\end{itemize}
}

\sheet{\Huge GARCH モデルの理論(1)}

{\huge
\begin{description}
	\item[{\Huge \epsilon_t}\ :\ ] 数式上は実数値離散時間確率過程を表すとする.応用上は金融資産のリターンの履歴を表す.
	\item[{\Huge \psi_t}\ :\ ] 時点 $t$ までの情報構造.時点 $t$ までの過程の $\sigma$ -集合を表す.\\
		$\epsilon_t\ は時点\ t\ での確率変数であるから,それぞれの時点で何らかの値を取る.これを\ \epsilon_t(\omega_t)\ と表記すれば,$
		$時点を\ t_0,\ t_1,\ \cdots,t_n\ として$
		\[
			\left\{\vector{\omega}\ |\ \vector{\omega} \equiv (\omega_0,\ \omega_1,\ \cdots,\omega_n) \right\}.
		\]
		$この集合を含む最小の \sigma 加法族が即ち\ \psi_t\ である.$
\end{description}
$\epsilon_t$ は,時点 $t-1$ までの履歴の情報 $\psi_{t-1}$ が与えられた下で, 正規分布 $N(0, h_t)$ に従うとする.(正規分布でなくてもよい)
\[
	\epsilon_t|\psi_{t-1} \sim N(0,\ h_t).
\]
$GARCH$ モデルの式は以下の様に表される.
\[
	h_t = \alpha_0 + \sum_{i=1}^{q} \alpha_i \epsilon_{t-i}^2 + \sum_{i=1}^{p} \beta_i h_{t-i}.
\]
}

\sheet{\Huge GARCH モデルの理論(2)}

{\huge
$GARCH$ モデルは金融資産のリターンの履歴 $\epsilon$ とリターンの分散 $h$ を使ったモデルである.
\[
	h_t = \alpha_0 + \sum_{i=1}^{q} \alpha_i \epsilon_{t-i}^2 + \sum_{i=1}^{p} \beta_i h_{t-i}.
\]
\begin{eqnarray*}
	&& p,\ q はどの時点までの履歴をモデルに入れるかを表す,\\
	&& \alpha_0 \geq 0, \hspace{20pt} \alpha_i > 0, \\
	&& \beta_i \geq 0.
\end{eqnarray*}
係数の非負性は,分散 $h_t$ が負数にならない為にある.
}


\end{document}