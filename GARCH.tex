\documentclass[slide,10pt]{jsarticle}
\usepackage{amsmath}
\usepackage{amssymb}
\usepackage{amsfonts}
\usepackage{latexsym}
\usepackage{ascmac}
\usepackage[dvipdfmx]{color}
\usepackage{pxfonts}
\def\qed{\hfill $\Box$}
\def\vector#1{\mbox{\boldmath $#1$}}
\def\sheet #1{\section*{\centering \large \bfseries #1}}

\begin{document}

\title{\fontsize{45pt}{2cm}\selectfont \\ゼミ発表\\GARCH モデル\\}
\author{\Huge 学籍番号:201311324 百合川尚学}
\date{\Huge 2016/09/19}
\maketitle

\sheet{\Huge GARCH モデルとは何か}

{\huge
\begin{itemize}
	\vspace{5mm}
	\item $Generalized\ Autoregressive\ Conditional\ Heteroskedastic\ $の頭文字をとったもの.
	\vspace{5mm}
	\item 直訳すると''一般化された 自己回帰 条件付き 不均一分散''の確率過程.
	\vspace{5mm}
	\item $Tim\ Bollerslev$教授によって考案されたモデル.
	\vspace{5mm}
	\item 例えば,$Black-Scholes$のオプション価格方程式はサンプルデータが一定の平均と分散を持つという強い仮定に基づくが,
		$GARCH$モデルは金融資産の分散の不均一であるという場合にも対応出来る.
	\vspace{5mm}
	\item 金融資産の価値の変動の履歴が確率過程に従っているとして,$GARCH$モデル式はその確率過程を表す.
	\vspace{5mm}
	\item モデル式の中に資産価値の変動の分散の履歴を組み入れている.
	\vspace{5mm}
	\item $S\&P500\ index\ option$ のサンプルデータからパラメータを求めて,インデックスのサンプル外の部分を計算したところ,
		他のモデルに比べて実際の値との差を小さくできた $Heston,\ Nandi (2000)$.
	\vspace{5mm}
	\item $GARCH$モデルがサンプル外の値を上手く言い当てたのは,インデックスの履歴とインデックスのボラティリティとの相関関係を活用できるからである $Heston,\ Nandi (2000)$.
\end{itemize}
}

\sheet{\Huge GARCH モデルの理論(1)}

{\huge
\begin{description}
	\item[{\Huge \epsilon_t}\ :\ ] 数式上は実数値離散時間確率過程を表すとする.応用上は金融資産のリターンの履歴を表す.
	\item[{\Huge \psi_t}\ :\ ] 時点 $t$ までの情報構造.時点 $t$ までの過程の $\sigma$ -集合を表す.\\
		$\epsilon_t\ は時点\ t\ での確率変数であるから,それぞれの時点で何らかの値を取る.これを\ \epsilon_t(\omega_t)\ と表記すれば,$
		$時点を\ t_0,\ t_1,\ \cdots,t_n\ として$
		\[
			\left\{\vector{\omega}\ |\ \vector{\omega} \equiv (\omega_0,\ \omega_1,\ \cdots,\omega_n) \right\}.
		\]
		$この集合を含む最小の \sigma 加法族が即ち\ \psi_t\ である.$
\end{description}
$\epsilon_t$ は,時点 $t-1$ までの履歴の情報 $\psi_{t-1}$ が与えられた下で, 正規分布 $N(0, h_t)$ に従うとする.(正規分布でなくてもよい)
\[
	\epsilon_t|\psi_{t-1} \sim N(0,\ h_t).
\]
$GARCH$ モデルの式は以下の様に表される: \\
\qquad 時点 t でのボラティリティ $h_t$ について,
\[
	h_t = \alpha_0 + \sum_{i=1}^{q} \alpha_i \epsilon_{t-i}^2 + \sum_{i=1}^{p} \beta_i h_{t-i}.
\]
}

\sheet{\Huge GARCH モデルの理論(2) 定常性}

{\huge
$GARCH$ モデルは金融資産のリターンの履歴 $\epsilon$ とリターンの分散 $h$ を使ったモデルである.
\[
	h_t = \alpha_0 + \sum_{i=1}^{q} \alpha_i \epsilon_{t-i}^2 + \sum_{i=1}^{p} \beta_i h_{t-i}.
\]
\begin{eqnarray*}
	&& p,\ q はどの時点までの履歴をモデルに入れるかを表す,\\
	&& \alpha_0 \geq 0, \hspace{20pt} \alpha_i > 0, \hspace{20pt} \beta_i \geq 0.
\end{eqnarray*}
係数の非負性は,分散 $h_t$ が負数にならない為にある.\\[1ex]
$GARCH(p,\ q)$で表されるリターンの確率過程$\{ \epsilon_t \}_t$が広義定常,すなわち,時点に依存しない値が存在して,
\begin{align*}
	E[\epsilon_t] &= \mu < \infty, \\
	V[\epsilon_t] &= \sigma^2 < \infty, \\
	Cov[\epsilon_t,\ \epsilon_{t-i}] &= \rho_i < \infty.
\end{align*}
と表せるための必要十分条件は
\[
	\sum_{i=1}^{q} \alpha_i + \sum_{i=1}^{p} \beta_i < 1
\]
が成り立つことであり,
\[
	\epsilon_t \equiv \eta_t h_t^{\frac{1}{2}}, \hspace{20pt} \eta_t \sim N(0,\ 1)\ i.i.d.
\]
を仮定すれば,
\[
	E[\epsilon_t] = 0,\ V[\epsilon_t] = \alpha_0 \left( 1 - \sum_{i=1}^{q} \alpha_i - \sum_{i=1}^{p} \beta_i \right)^{-1},
	\ Cov[\epsilon_t,\ \epsilon_s] &= 0
\]
が成り立つ.特に,$GARCH(1,\ 1)$では

\[
	\alpha_1 + \beta_1 < 1 \hspace{20pt} \Leftrightarrow \hspace{20pt}
	\begin{cases}
		E[\epsilon_t] &= 0 \\
		V[\epsilon_t] &= \alpha_0 \left( 1 - \alpha_1 - \beta_1 \right)^{-1} \\
		Cov[\epsilon_t,\ \epsilon_s] &= 0
	\end{cases}
\]

である.
}

\sheet{\Huge GARCH モデルの理論(3) 2m次モーメントの存在}

{\huge
もう一つ重要な定理があり,それは次のものである: \\
\vspace{10mm}
$GARCH(1,\ 1)$ モデルについて,モデル式
\[
	h_t = \alpha_0 + \alpha_1 \epsilon_{t-1}^2 + \beta_1 h_{t-1}.
\]
に従う確率過程 $\{ \epsilon_t \}_t$ の$2m$次のモーメント$E[\epsilon_t^{2m}]$が存在する為の必要十分条件は,
\[
	\sum_{j=0}^{m} \binom{m}{j} a_j \alpha_1^j \beta_1^{m-j} < 1, \hspace{20pt} a_0 = 1,\ a_j = \prod_{i=1}^{j} (2i-1)
\]
が成り立つことである.

}

\sheet{\Huge References}
{\huge

\begin{itemize}
	\vspace{10mm}
	\item [1].\ Tim\ Bollerslev\ (1986),\quad ''GENERALIZED\ AUTOREGRESSIVE\ CONDITIONAL\ HETEROSKEDASTICITY''.
	\vspace{5mm}
	\item [2].\ Steven\ L.\ Heston,\ Saikat\ Nandi\ (2000),\quad ''A\ Closed-Form\ GARCH\ Option\ Valuation\ Model''.
\end{itemize}

}

\end{document}